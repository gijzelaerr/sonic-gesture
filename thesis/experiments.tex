\chapter{Experiments}
\label{ch:experiments}


\section{The dataset}
A dataset was created to perform the experiments. In total there are 74 movies containing 20 different test subjects performing the complete sequence of 28 hand symbols. The movies where recorded with Sonic Gesture, with a resolution of 532x400 and a frame rate of 10. The test subjects where recorded while looking at a computer screen and asked to mimic the examples as in figure\ref{fig:hands_mirrored}, \ref{fig:hands_normal} and \ref{fig:hands_extra}. The movies with 12 test subjects where recorded with a simple (almost empty and smooth) background, 3 where recorded with a complex background and 6 where recorded with the same complex background but also with a poster with skinlike colors.

\begin{table}
\begin{tabular}{llll}
	Test Subject & Movies & Background & Notes \\
	\hline
	Anne     & 5 & Simple & None \\
	Arjan    & 5 & Simple & None \\
	Gijs     & 5 & Simple & None \\
	Ivo      & 5 & Simple & None \\
	Jasper 1 & 5 & Simple & None \\
	Peter    & 5 & Simple & None \\
	Hanne    & 5 & Simple & None \\

	Jasper 2 & 3 & Simple & None \\
	Ork      & 3 & Simple & None \\
	Roberto  & 3 & Simple & None \\
	Xiaong   & 3 & Simple & None \\
			
	Gosia    & 3 & Complex & None \\
	Hamdi    & 3 & Complex & None \\
	Michael  & 3 & Complex & None \\
	
	Bas      & 3 & Complex + skin-like colors & None \\
	Koen     & 3 & Complex + skin-like colors & None \\
	Sil      & 3 & Complex + skin-like colors & None \\
	Victoria & 3 & Complex + skin-like colors & None \\
	Chu      & 3 & Complex + skin-like colors & Quite yellow skin color, similar to brick wall \\
	Stratos  & 3 & Complex + skin-like colors & Quite pale skin color, similar to white wall \\
\end{tabular}
\caption{Dataset details}
\end{table}

The dataset is manually labeled. For every sequence of frames where a hand pose is performed a frame number is labeled where the pose resembles the example the most. Figure\ref{fig:gijs5} is an example of the labeled frames.

Usually the 12 Curwen solfege hand symbols are performed in front of the torso as shown in Figure~\ref{fig:hands_normal}. To increase the number of hand poses to be recogniced, all 12 Curwen are also performed mirrored next to the body of the recorded subject, see Figure~\ref{fig:hands_mirrored}. Additionally four extra hand symbols have been added that are not part of the Curwen sequence, see Figure~\ref{fig:hands_extra}. These last four symbols are performed next to the head.

\section{Evaluation}


\section{Implementation}
OpenCV, QT, CMake, Liblo, open source, website

\section{Results}


\section{Images}

\renewcommand{\thesubfigure}{\thefigure.\roman{subfigure}} 
\begin{figure}[htbp]
\begin{center}
\subfloat[Do]{\label{fig:hand_0}\includegraphics[width=0.2\linewidth,height=0.15\linewidth]{figures/examples/0.jpg}}
\hspace{0.03\linewidth}
\subfloat[Di]{\label{fig:hand_1}\includegraphics[width=0.2\linewidth,height=0.15\linewidth]{figures/examples/1.jpg}}
\hspace{0.03\linewidth}
\subfloat[Re]{\label{fig:hand_2}\includegraphics[width=0.2\linewidth,height=0.15\linewidth]{figures/examples/2.jpg}}
\hspace{0.03\linewidth}
\subfloat[Ri]{\label{fig:hand_3}\includegraphics[width=0.2\linewidth,height=0.15\linewidth]{figures/examples/3.jpg}}
\hspace{0.03\linewidth}
\subfloat[Mi]{\label{fig:hand_4}\includegraphics[width=0.2\linewidth,height=0.15\linewidth]{figures/examples/4.jpg}}
\hspace{0.03\linewidth}
\subfloat[Fa]{\label{fig:hand_5}\includegraphics[width=0.2\linewidth,height=0.15\linewidth]{figures/examples/5.jpg}}
\hspace{0.03\linewidth}
\subfloat[Fi]{\label{fig:hand_6}\includegraphics[width=0.2\linewidth,height=0.15\linewidth]{figures/examples/6.jpg}}
\hspace{0.03\linewidth}
\subfloat[Sol]{\label{fig:hand_7}\includegraphics[width=0.2\linewidth,height=0.15\linewidth]{figures/examples/7.jpg}}
\hspace{0.03\linewidth}
\subfloat[Si]{\label{fig:hand_8}\includegraphics[width=0.2\linewidth,height=0.15\linewidth]{figures/examples/8.jpg}}
\hspace{0.03\linewidth}
\subfloat[La]{\label{fig:hand_9}\includegraphics[width=0.2\linewidth,height=0.15\linewidth]{figures/examples/9.jpg}}
\hspace{0.03\linewidth}
\subfloat[Li]{\label{fig:hand_10}\includegraphics[width=0.2\linewidth,height=0.15\linewidth]{figures/examples/10.jpg}}
\hspace{0.03\linewidth}
\subfloat[Ti]{\label{fig:hand_11}\includegraphics[width=0.2\linewidth,height=0.15\linewidth]{figures/examples/11.jpg}}
\end{center}
\caption{All mirrored Curwen hand poses}
\label{fig:hands_mirrored}
\end{figure}

\begin{figure}[htbp]
\begin{center}
\subfloat[Do]{\label{fig:hand_12}\includegraphics[width=0.2\linewidth,height=0.15\linewidth]{figures/examples/12.jpg}}
\hspace{0.03\linewidth}
\subfloat[Di]{\label{fig:hand_13}\includegraphics[width=0.2\linewidth,height=0.15\linewidth]{figures/examples/13.jpg}}
\hspace{0.03\linewidth}
\subfloat[Re]{\label{fig:hand_14}\includegraphics[width=0.2\linewidth,height=0.15\linewidth]{figures/examples/14.jpg}}
\hspace{0.03\linewidth}
\subfloat[Ri]{\label{fig:hand_15}\includegraphics[width=0.2\linewidth,height=0.15\linewidth]{figures/examples/15.jpg}}
\hspace{0.03\linewidth}
\subfloat[Mi]{\label{fig:hand_16}\includegraphics[width=0.2\linewidth,height=0.15\linewidth]{figures/examples/16.jpg}}
\hspace{0.03\linewidth}
\subfloat[Fa]{\label{fig:hand_17}\includegraphics[width=0.2\linewidth,height=0.15\linewidth]{figures/examples/17.jpg}}
\hspace{0.03\linewidth}
\subfloat[Fi]{\label{fig:hand_18}\includegraphics[width=0.2\linewidth,height=0.15\linewidth]{figures/examples/18.jpg}}
\hspace{0.03\linewidth}
\subfloat[Sol]{\label{fig:hand_19}\includegraphics[width=0.2\linewidth,height=0.15\linewidth]{figures/examples/19.jpg}}
\hspace{0.03\linewidth}
\subfloat[Si]{\label{fig:hand_20}\includegraphics[width=0.2\linewidth,height=0.15\linewidth]{figures/examples/20.jpg}}
\hspace{0.03\linewidth}
\subfloat[La]{\label{fig:hand_21}\includegraphics[width=0.2\linewidth,height=0.15\linewidth]{figures/examples/21.jpg}}
\hspace{0.03\linewidth}
\subfloat[Li]{\label{fig:hand_22}\includegraphics[width=0.2\linewidth,height=0.15\linewidth]{figures/examples/22.jpg}}
\hspace{0.03\linewidth}
\subfloat[Ti]{\label{fig:hand_23}\includegraphics[width=0.2\linewidth,height=0.15\linewidth]{figures/examples/23.jpg}}
\end{center}
\caption{All Curwen hand poses}
\label{fig:hands_normal}
\end{figure}


\begin{figure}[htbp]
\begin{center}
\subfloat[Extra1]{\label{fig:hand_24}\includegraphics[width=0.2\linewidth,height=0.15\linewidth]{figures/examples/24.jpg}}
\hspace{0.03\linewidth}
\subfloat[Extra2]{\label{fig:hand_25}\includegraphics[width=0.2\linewidth,height=0.15\linewidth]{figures/examples/25.jpg}}
\hspace{0.03\linewidth}
\subfloat[Extra3]{\label{fig:hand_26}\includegraphics[width=0.2\linewidth,height=0.15\linewidth]{figures/examples/26.jpg}}
\hspace{0.03\linewidth}
\subfloat[Extra4]{\label{fig:hand_27}\includegraphics[width=0.2\linewidth,height=0.15\linewidth]{figures/examples/27.jpg}}
\end{center}
\caption{The extra hand poses}
\label{fig:hands_extra}
\end{figure}


\begin{figure}[htbp]
\begin{center}
\subfloat[Do]{\label{fig:gijs5_0}\includegraphics[width=0.2\linewidth,height=0.15\linewidth]{figures/gijs5/0.png}}
\hspace{0.03\linewidth}
\subfloat[Di]{\label{fig:gijs5_1}\includegraphics[width=0.2\linewidth,height=0.15\linewidth]{figures/gijs5/1.png}}
\hspace{0.03\linewidth}
\subfloat[Re]{\label{fig:gijs5_2}\includegraphics[width=0.2\linewidth,height=0.15\linewidth]{figures/gijs5/2.png}}
\hspace{0.03\linewidth}
\subfloat[Ri]{\label{fig:gijs5_3}\includegraphics[width=0.2\linewidth,height=0.15\linewidth]{figures/gijs5/3.png}}
\hspace{0.03\linewidth}
\subfloat[Mi]{\label{fig:gijs5_4}\includegraphics[width=0.2\linewidth,height=0.15\linewidth]{figures/gijs5/4.png}}
\hspace{0.03\linewidth}
\subfloat[Fa]{\label{fig:gijs5_5}\includegraphics[width=0.2\linewidth,height=0.15\linewidth]{figures/gijs5/5.png}}
\hspace{0.03\linewidth}
\subfloat[Fi]{\label{fig:gijs5_6}\includegraphics[width=0.2\linewidth,height=0.15\linewidth]{figures/gijs5/6.png}}
\hspace{0.03\linewidth}
\subfloat[Sol]{\label{fig:gijs5_7}\includegraphics[width=0.2\linewidth,height=0.15\linewidth]{figures/gijs5/7.png}}
\hspace{0.03\linewidth}
\subfloat[Si]{\label{fig:gijs5_8}\includegraphics[width=0.2\linewidth,height=0.15\linewidth]{figures/gijs5/8.png}}
\hspace{0.03\linewidth}
\subfloat[La]{\label{fig:gijs5_9}\includegraphics[width=0.2\linewidth,height=0.15\linewidth]{figures/gijs5/9.png}}
\hspace{0.03\linewidth}
\subfloat[Li]{\label{fig:gijs5_10}\includegraphics[width=0.2\linewidth,height=0.15\linewidth]{figures/gijs5/10.png}}
\hspace{0.03\linewidth}
\subfloat[Ti]{\label{fig:gijs5_11}\includegraphics[width=0.2\linewidth,height=0.15\linewidth]{figures/gijs5/11.png}}
\hspace{0.03\linewidth}
\subfloat[Do]{\label{fig:gijs5_12}\includegraphics[width=0.2\linewidth,height=0.15\linewidth]{figures/gijs5/12.png}}
\hspace{0.03\linewidth}
\subfloat[Di]{\label{fig:gijs5_13}\includegraphics[width=0.2\linewidth,height=0.15\linewidth]{figures/gijs5/13.png}}
\hspace{0.03\linewidth}
\subfloat[Re]{\label{fig:gijs5_14}\includegraphics[width=0.2\linewidth,height=0.15\linewidth]{figures/gijs5/14.png}}
\hspace{0.03\linewidth}
\subfloat[Ri]{\label{fig:gijs5_15}\includegraphics[width=0.2\linewidth,height=0.15\linewidth]{figures/gijs5/15.png}}
\hspace{0.03\linewidth}
\subfloat[Mi]{\label{fig:gijs5_16}\includegraphics[width=0.2\linewidth,height=0.15\linewidth]{figures/gijs5/16.png}}
\hspace{0.03\linewidth}
\subfloat[Fa]{\label{fig:gijs5_17}\includegraphics[width=0.2\linewidth,height=0.15\linewidth]{figures/gijs5/17.png}}
\hspace{0.03\linewidth}
\subfloat[Fi]{\label{fig:gijs5_18}\includegraphics[width=0.2\linewidth,height=0.15\linewidth]{figures/gijs5/18.png}}
\hspace{0.03\linewidth}
\subfloat[Sol]{\label{fig:gijs5_19}\includegraphics[width=0.2\linewidth,height=0.15\linewidth]{figures/gijs5/19.png}}
\hspace{0.03\linewidth}
\subfloat[Si]{\label{fig:gijs5_20}\includegraphics[width=0.2\linewidth,height=0.15\linewidth]{figures/gijs5/20.png}}
\hspace{0.03\linewidth}
\subfloat[La]{\label{fig:gijs5_21}\includegraphics[width=0.2\linewidth,height=0.15\linewidth]{figures/gijs5/21.png}}
\hspace{0.03\linewidth}
\subfloat[Li]{\label{fig:gijs5_22}\includegraphics[width=0.2\linewidth,height=0.15\linewidth]{figures/gijs5/22.png}}
\hspace{0.03\linewidth}
\subfloat[Ti]{\label{fig:gijs5_23}\includegraphics[width=0.2\linewidth,height=0.15\linewidth]{figures/gijs5/23.png}}
\hspace{0.03\linewidth}
\subfloat[Extra1]{\label{fig:gijs5_24}\includegraphics[width=0.2\linewidth,height=0.15\linewidth]{figures/gijs5/24.png}}
\hspace{0.03\linewidth}
\subfloat[Extra2]{\label{fig:gijs5_25}\includegraphics[width=0.2\linewidth,height=0.15\linewidth]{figures/gijs5/25.png}}
\hspace{0.03\linewidth}
\subfloat[Extra3]{\label{fig:gijs5_26}\includegraphics[width=0.2\linewidth,height=0.15\linewidth]{figures/gijs5/26.png}}
\hspace{0.03\linewidth}
\subfloat[Extra4]{\label{fig:gijs5_27}\includegraphics[width=0.2\linewidth,height=0.15\linewidth]{figures/gijs5/27.png}}
\end{center}
\caption{Stills from recording number 5, test subject 'gijs'}
\label{fig:gijs5}
\end{figure}


